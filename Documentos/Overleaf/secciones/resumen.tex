\chapter*{Resumen} \label{chp:abstract}

Este Trabajo de Fin de Grado explora el uso de herramientas del Análisis Topológico de Datos (TDA) para el estudio de trayectorias espaciales en contextos de baja y alta dimensión. A través de técnicas como la homología persistente y el algoritmo Mapper, se extraen características estructurales de trayectorias GPS, permitiendo identificar patrones de conectividad, ciclos y componentes relevantes que escapan a los métodos clásicos de análisis. El trabajo incluye una implementación práctica utilizando la biblioteca Giotto-TDA en Python, aplicada sobre el conjunto de datos Geolife. Además, se comparan los resultados topológicos con enfoques tradicionales como PCA, t-SNE y algoritmos de agrupamiento como DBSCAN y KMeans.

Los resultados demuestran que el enfoque topológico permite detectar estructuras globales robustas, como bucles o trayectorias anómalas, con mayor sensibilidad que las técnicas estadísticas convencionales. Asimismo, se destacan las ventajas del uso de TDA en entornos ruidosos o con alta complejidad geométrica. Se discuten también las implicaciones de este análisis para aplicaciones en movilidad urbana, transporte aéreo y sistemas dinámicos, así como su contribución a objetivos de desarrollo sostenible. Finalmente, se proponen líneas futuras de investigación orientadas a integrar TDA con aprendizaje automático profundo y explorar nuevas representaciones vectoriales topológicas.

\textbf{Palabras Clave}: Análisis Topológico de Datos, Homología Persistente, Mapper, Trayectorias GPS, Clustering, PCA, t-SNE, Geolife.

%%%%%%%%%%%%%%%%%%%%%%%%%%%%%%%%%%%%%%%%%%%%%%%%%%%%%%%%%%%%%%%%%%%%%%%%%%%%%%%%

\newpage

%%%%%%%%%%%%%%%%%%%%%%%%%%%%%%%%%%%%%%%%%%%%%%%%%%%%%%%%%%%%%%%%%%%%%%%%%%%%%%%%

\chapter*{Abstract}

This Bachelor Thesis investigates the application of Topological Data Analysis (TDA) techniques to the study of spatial trajectories in both low and high-dimensional contexts. Using tools such as persistent homology and the Mapper algorithm, the project extracts structural features from GPS trajectory data, enabling the identification of connectivity patterns, loops, and relevant components that are often missed by classical methods. The practical implementation is based on the Giotto-TDA Python library and is applied to the Geolife dataset. Results from TDA are compared to traditional approaches such as PCA, t-SNE, and clustering algorithms like DBSCAN and KMeans.

The findings show that topological methods offer a more sensitive detection of robust global structures—such as loops or outlier trajectories—than conventional statistical techniques. TDA proves particularly advantageous in noisy or geometrically complex settings. The work discusses its relevance for applications in urban mobility, air traffic, and dynamic systems, as well as its alignment with Sustainable Development Goals. Future research directions are proposed, including integration with deep learning models and the exploration of novel topological vector representations.

\textbf{Keywords}: Topological Data Analysis, Persistent Homology, Mapper, GPS Trajectories, Clustering, PCA, t-SNE, Geolife.
