

\chapter{Introducción} \label{chp:intro}


\section{Motivación del proyecto} \label{sct:intro:motivacion}

El desarrollo de este Trabajo Fin de Grado surge de la necesidad actual de contar con metodologías robustas que permitan extraer información cualitativa y cuantitativa a partir de datos complejos. El análisis topológico y estadístico se presenta como una herramienta innovadora para la interpretación de fenómenos en ámbitos tan diversos como la movilidad urbana (por ejemplo, seguimiento de vehículos y dispositivos móviles) o procesos abstractos de optimización y sistemas dinámicos (como el estudio de atractores en modelos caóticos). La capacidad de capturar la "forma" subyacente de los datos mediante técnicas de análisis topológico de datos (TDA) y combinarlas con métodos de clustering y análisis estadístico, justifica el objetivo de este proyecto, ofreciendo potenciales aplicaciones en inteligencia artificial, robótica y análisis de sistemas complejos.

%%%%%%%%%%%%%%%%%%%%%%%%%%%%%%%%%%%%%%%%%%%%%%%%%%%%%%%%%%%%%%%%%%%%%%%%%%%%%%%% %%%%%%%%%%%%%%%%%%%%%%%%%%%%%%%%%%%%%%%%%%%%%%%%%%%%%%%%%%%%%%%%%%%%%%%%%%%%%%%%

\section{Contexto del proyecto} \label{sct:intro:contexto}

Este Trabajo de Fin de Grado se enmarca en el emergente campo del análisis topológico de datos, que ha ganado relevancia en los últimos años gracias a su eficacia para detectar estructuras y patrones en datos de alta complejidad. Diversos estudios y avances hechos en los últimos años han demostrado el valor de aplicar herramientas topológicas para el análisis de trayectorias, tanto en contextos físicos como en aplicaciones más abstractas \cite{EsteveFalco2024} . La revisión del estado del arte evidencia la utilidad de integrar enfoques teóricos –como la construcción de complejos simpliciales y el uso de homología persistente – con técnicas estadísticas y de clustering para lograr una interpretación integral de los datos. Asimismo, cabe destacar la relevancia de trabajos previos que han establecido las bases para la modelización y análisis de series temporales y datos funcionales \cite{espinoza2022tda_air}, proporcionando el contexto necesario para este proyecto.

%%%%%%%%%%%%%%%%%%%%%%%%%%%%%%%%%%%%%%%%%%%%%%%%%%%%%%%%%%%%%%%%%%%%%%%%%%%%%%%% %%%%%%%%%%%%%%%%%%%%%%%%%%%%%%%%%%%%%%%%%%%%%%%%%%%%%%%%%%%%%%%%%%%%%%%%%%%%%%%%
\vspace{2cm}

\section{Objetivos} \label{sct:intro:objetivos}

El objetivo principal de este Trabajo Fin de Grado es desarrollar una metodología integral para el análisis topológico y estadístico de trayectorias en espacios de baja y alta dimensión, series temporales multivariantes, que permita extraer información relevante para la interpretación y clasificación de dichos datos.

\begin{itemize}


\item[•] \textbf{Definir y documentar el problema:} Delimitar las variables y estructuras de datos que representan las trayectorias, tanto en escenarios físicos (movilidad, dispositivos, etc.) como en contextos abstractos (modelos dinámicos y optimización).

\item[•] \textbf{Diseñar la metodología de análisis:} Integrar técnicas de TDA para extraer características topológicas con métodos estadísticos y algoritmos de clustering que faciliten la identificación de patrones y agrupaciones en los datos.

\item[•] \textbf{Implementar y evaluar la propuesta:} Realizar un estudio experimental que valide la eficacia de la metodología, evaluando aspectos como la robustez frente al ruido, la sensibilidad de los algoritmos y la complejidad computacional.

\item[•] \textbf{Interpretar y sintetizar resultados:} Analizar los resultados obtenidos para ofrecer conclusiones que aporten al conocimiento en el análisis de trayectorias y que puedan abrir nuevas líneas de investigación en el campo.
\end{itemize}

%%%%%%%%%%%%%%%%%%%%%%%%%%%%%%%%%%%%%%%%%%%%%%%%%%%%%%%%%%%%%%%%%%%%%%%%%%%%%%%% %%%%%%%%%%%%%%%%%%%%%%%%%%%%%%%%%%%%%%%%%%%%%%%%%%%%%%%%%%%%%%%%%%%%%%%%%%%%%%%%

\section{Estructura del Documento} \label{sct:intro_estructura}

Este Trabajo Fin de Grado se organiza en diversos capítulos, cada uno de ellos enfocado en aspectos específicos del desarrollo del proyecto. A continuación, se describe brevemente el contenido de cada capítulo

\begin{itemize} \item \textbf{Capítulo 1: Introducción.} Se expone la motivación para realizar el proyecto, el contexto en el que se encuentra, los objetivos del mismo y una breve descripción de la estructura del documento.

\item \textbf{Capítulo 2: Trabajo relacionado y Estado del
Arte.} Se realiza una revisión de la literatura y se presentan los conceptos y herramientas fundamentales relacionados con TDA, modelización de trayectorias y técnicas de clustering, contextualizando el trabajo en el marco de estudios previos.

\item \textbf{Capítulo 3: Análisis Topológico de trayectorias en alta y baja dimensión} Se detalla el proceso de elección, preprocesamiento y modelización de datos, y se describe la metodología propuesta para la extracción y análisis de características topológicas de las trayectorias, así como la implementación de la propuesta, el desarrollo de algoritmos específicos y la realización de estudios experimentales para evaluar la eficacia del enfoque. 

\item \textbf{Capítulo 4: Resultados de TDA sobre trayectorias} Se presentan y analizan los resultados obtenidos durante la realización del trabajo, realizando comparativas con métodos existentes y discutiendo las implicaciones practicas y teóricas del trabajo.

\vspace{1cm}
\item \textbf{Capítulo 5: Conclusiones} Se evalúan los objetivos declarados en el trabajo, se proponen futuras líneas de investigación que podrían extender los hallazgos obtenidos, se exponen conclusiones personales del autor sobre el trabajo realizado, se identifican las limitaciones y se realiza un análisis del impacto del trabajo, y las decisiones tomadas a lo largo del mismo, que tienen como base el impacto y el potencial impacto respecto a los Objetivos de Desarrollo Sostenible (ODS) de la Agenda 2030 que sean relevantes para el trabajo realizado \cite{ODS} .

\item \textbf{Anexos y Bibliografía.} Se incluye material complementario, detalles técnicos y la documentación de las referencias utilizadas a lo largo del proyecto.
\end{itemize}

%%%%%%%%%%%%%%%%%%%%%%%%%%%%%%%%%%%%%%%%%%%%%%%%%%%%%%%%%%%%%%%%%%%%%%%%%%%%%%%% %%%%%%%%%%%%%%%%%%%%%%%%%%%%%%%%%%%%%%%%%%%%%%%%%%%%%%%%%%%%%%%%%%%%%%%%%%%%%%%%

Esta estructura permite una presentación clara y ordenada de todo el proceso de análisis topológico de trayectorias, facilitando la comprensión de la metodología, los resultados y la relevancia del trabajo dentro del campo de la inteligencia computacional y el análisis de datos.


%%%%%%%%%%%%%%%%%%%%%%%%%%%%%%%%%%%%%%%%%%%%%%%%%%%%%%%%%%%%%%%%%%%%%%%%%%%%%%%%
%%%%%%%%%%%%%%%%%%%%%%%%%%%%%%%%%%%%%%%%%%%%%%%%%%%%%%%%%%%%%%%%%%%%%%%%%%%%%%%%