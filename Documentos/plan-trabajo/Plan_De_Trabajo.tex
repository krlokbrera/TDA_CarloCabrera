%%-----------------------------------------------
%% Cargar datos relativos al TFG:
%% (actualizar estos datos en secciones/_DatosTFG.tex)
%%%%%%%%%%%%%%%%%%%%%%%%%%%%%%%%%
%% Información para la portada %%
%%%%%%%%%%%%%%%%%%%%%%%%%%%%%%%%%

%% Escribe Nombre y Apellidos del autor del trabajo:
\newcommand{\NombreAutor}{ Carlo Alessandro Cabrera }

%% Escribe el Grado: 
\newcommand{\Grado}{ Ingeniería Informática }

%% Escribe el Título del Trabajo:
\newcommand{\TituloTFG}{ Análisis topológico de trayectorias en baja y alta dimensión } 

%% Escribe Nombre y Apellidos del Tutor del trabajo: 
\newcommand{\NombreTutor}{ Juan Antonio Fernández del Pozo } 

% Escribe el Departamento al que pertenece el Tutor:
\newcommand{\Departamento}{ Departamento de Inteligencia Artificial (DIA) }

% Escribe la fecha de lectura, en formato: Mes - Año
\newcommand{\Fecha}{ JUNIO - 2025 }
%%***********************************************


%%-----------------------------------------------
%% Importar Preámbulo:
\input{include/Preambulo}

% Aquí puedes cambiar el titulo del Plan de Trabajo
% y añadir el titulo de tu TFG, o dejar solo "Plan de Trabajo"
\renewcommand{\TituloTFG}{ Plan de Trabajo}

%%-----------------------------------------------
%% Documento
\begin{document}
\input{include/Portada}

%%-----------------------------------------------
\chapter*{Descripción y Objetivos}

Este Trabajo Fin de Grado se centra en el análisis topológico y estadístico de trayectorias en espacios de baja y alta dimensión. Se aborda el estudio de datos que representan trayectorias, ya sean de sentido físico (movimiento de vehículos en zonas urbanas) o trazas derivadas de procesos de optimización y procesos aleatorios en espacios discretos o continuos, en este caso datos generados a partir del atractor de Lorenz. La propuesta tiene como fin identificar características topológicas relevantes en la nube de datos, lo que permitirá la aplicación de técnicas de clustering y clasificación para extraer conocimiento y comprender mejor el comportamiento de las trayectorias.

Los principales objetivos de este Trabajo de Fin de Grado son: 
\begin{itemize} 
\item[•] Desarrollar una metodología para extraer conocimiento de datos representados como trayectorias, aplicable tanto a escenarios físicos como a contextos abstractos. \item[•] Diseñar y validar una propuesta de análisis que abarque la adquisición de datos, su modelización, exploración, inferencia y síntesis de conclusiones. 
\item[•] Definir y documentar las variables y estructuras de datos de interés, estableciendo criterios claros para la interpretación y agrupación de trayectorias mediante técnicas de clustering. 
\item[•] Evaluar el rendimiento, la robustez (sensibilidad) y la complejidad computacional de las soluciones propuestas. 
\end{itemize}

\chapter*{Tareas}

El desarrollo de este Trabajo Fin de Grado se ha planificado mediante una serie de tareas estructuradas, cuyo seguimiento se facilita mediante un diagrama de Gantt que distribuye las actividades a lo largo de las 17 semanas de duración estimada del proyecto. Esta planificación garantiza una gestión eficiente del tiempo, abarcando desde la fase de investigación hasta la defensa del proyecto frente a un tribunal de evaluación. A continuación, se adjunta una lista de las tareas a realizar durante el desarrollo, así como un diagrama de Gantt para ayudar a la visualización de la distribución de las mismas.

\begin{enumerate}
    \item[T1 -] Investigacion Preliminar
    \item[T2 -] Elaboracion Plan de Trabajo
    \item[T3 -] Fundamentos y Estado del Arte
    \item[T4 -] Estudio del Problema (Modelo de Datos y Datos a analizar)
    \item[T5 -] Estudio de Soluciones
    \item[T6 -] Extraccion, Preparacion, Codificacion y Analisis
    \item[T7 -] Memoria de Seguimiento
    \item[T8 -] Documentacion de Resultados
    \item[T9 -] Ficha Resumen del TFG
    \item[T10 -] Memoria TFG
    \item[T11 -] Pres. Defensa
\end{enumerate}


\begin{landscape}
% \ganttbar[name=NAME]{NAME}{SEMANA_INICIO}{SEMANA_FIN} \\
% \ganttlink[link type={s-s || s-f || f-s || f-f}]{elem0}{elem1}
%   s-s (start-to-start) 
%   s-f (start-to-finish)
%   f-s (finish-to-start)
%   f-f (finish-to-finish)

% El TFG se desarrolla en, aproximadamente, 17-18 semanas
    \begin{ganttchart}[
            hgrid,
            vgrid,
            x unit=1cm,
            y unit chart=0.9cm
        ]{1}{17}

        % Definicion de cada tarea y las semanas de inicio y fin
        \gantttitle{Semana}{17} \\
        \gantttitlelist{1,...,17}{1} \\
         % Tareas del plan de trabajo:
        \ganttbar[name=T1]{Investigación Preliminar}{1}{1} \\
        \ganttbar[name=T2]{Elaboración del Plan de Trabajo}{2}{2} \\
        \ganttbar[name=T3]{Fundamentos y Estado del Arte}{3}{6} \\
        \ganttbar[name=T4]{Estudio del Problema (Modelo de Datos)}{3}{4} \\
        \ganttbar[name=T5]{Estudio de Soluciones (MTS, FDA)}{5}{7} \\
        \ganttbar[name=T6]{Extracción, Preparación, Codificación y Pruebas}{8}{14} \\
        \ganttbar[name=T7]{Memoria de Seguimiento}{8}{8} \\   
        \ganttbar[name=T8]{Documentación de Resultados}{15}{15} \\       
        \ganttbar[name=T9]{Memoria TFG}{3}{15} \\
        \ganttbar[name=T10]{Ficha Resumen del TFG}{16}{16} \\
        \ganttbar[name=T11]{Pres. Defensa}{16}{17}
        
        % Dependencias entre tareas:
        \ganttlink[link type=f-s]{T1}{T2}
        \ganttlink[link type=s-s]{T3}{T4}
        \ganttlink[link type=s-s]{T3}{T9}
        \ganttlink[link type=f-s]{T5}{T6}
        \ganttlink[link type=s-s]{T6}{T7}
        \ganttlink[link type=f-s]{T6}{T8}
        \ganttlink[link type= f-s]{T9}{T10}
        \ganttlink[link type= f-s]{T9}{T11}
    \end{ganttchart}

\end{landscape}


\chapter*{Propuesta de Trabajo Oficial}


\title{Trabajo Fin de Grado\\ Análisis topológico de trayectorias en baja y alta dimensión}

\author{\large Juan A. Fdez del Pozo\\
  \normalsize Computational Intelligence Group, Universidad Polit\'ecnica de Madrid\\
  \normalsize Campus de Montegancedo, Boadilla del Monte, 28660 Madrid, Spain
email{\{jafernandez\}@fi.upm.es}\\
\date{}}

\textheight        25cm
\textwidth         17.5cm
\evensidemargin    -0.25cm
\oddsidemargin =   -0.25cm
\topmargin         -3.0cm

%%dimensiones hxv de figuras
%\newcommand{\hsz}{7.0cm}
%\newcommand{\hsz}{15.0cm}
%\newcommand{\vsz}{0.75cm}

\section{Resumen general del trabajo}

El TFG propone hacer un análsis topológico y estadístico de trayectorias ($X(t) \sim F(R^n)$, $t \in R$), que pueden tener sentido fisico (vehículos, m\'oviles,\ldots) o ser trazas en espacios de soluciones de problemas de optimización o procesos aleatorios sobre un espacio de estados discreto o continuo.
La exploración de los datos considera la identificación de características topológicas, clustering o clasificación en la nube de datos (conjunto de trayectorias).

%Trabajo exclusivo del alumno: \ldots, \\
%del Grado de: \ldots \\
%N\'umero de Matr\'{\i}cula: \ldots

%% Section 2
\section{Lista de objetivos concretos}

\paragraph{Definición del problema:}

Se trata de desarrollar una metodología para extraer conocimiento de los datos, reales o sintéticos,
en términos de series temporales y/o de datos funcionales y
diseñar una propuesta de análisis de los datos:
adquisición de datos, modelización, exploración, inferencia y sintesis de conclusiones.\\

Documentar las variables y
establecer los objetivos para la propuesta de análisis.
Planificar el proceso de preparación de los datos, análisis y evaluación del resultado.

\paragraph{Análisis de requisitos:}

Definir las estructruras de datos para representar la información de interés.
Mostrar un descriptiva de los datos, generar una estructura de grupos de trayectorias interpretable.
Estudiar el rendimiento, robusted (sensibilidad) del análisis y la complejidad computacional.


%% Section 3
\section{Desglose de la dedicación total del trabajo en horas}
%% 11 ECTS, 297 horas, 10+20+30+40+20+90+40+47

\begin{enumerate}

\item  Redacción del plan de trabajo (10)

\item  Fundamentos, estado del arte, alineamiento con los ODS  (20)

\item  Estudio del problema: modelo de datos (trayectoria) y de datos funcionales (30)

\item  Estudio de algunas soluciones seg\'un el tipo de dato (MTS, FDA) que soporta las hipótesis (40)

\item  Redacci\'on de la memoria de seguimiento (20)

\item  Extracción de la muestra (adquisición de datos), Preparación,
Descriptiva y Exploración, Codificaci\'on de algoritmos y Pruebas (90)

\item  Documentaci\'on de los resultados del an\'alisis (40)

\item  Redacci\'on de la memoria final y preparaci\'on de la presentación del trabajo (47)

\end{enumerate}

%% Section 4
\section{Conocimientos previos recomendados para hacer el trabajo}

La teor\'{\i}a subyacente a los problemas que se quieren desarrollar es conocida por
los alumnos que han cursado las asignaturas del Grado.
El alumno podrá documentarse en análisis topológico de datos y herramientas software disponibles.

La programaci\'on y documentaci\'on debe realizarse en espa\~nol/ingl\'es,
el plan de trabajo y las memorias en espa\~nol/ingl\'es.

Es recomendable conocer a nivel b\'asico \LaTeX para preparar documentos: memoria y presentación.

El trabajo se puede realizar bajo Windows o Linux, en el entorno R, Python,\ldots.

%% Section 5
%\section{Fichero con los detalles del trabajo}

\begin{thebibliography}{90}

\bibitem{R3}
  RStudio \\
   {\tt https://rstudio.com/}

\bibitem{R5}
  LaTex \\
{\tt https://es.overleaf.com/}

\bibitem{R7}
  El arte de la estadística: cómo aprender de los datos \\
  Spiegelhalter, David J., Ed. Capitan Swing, 2023.\\
{\tt https://capitanswing.com/\-libros/el-arte-de-la-estadistica/}

\bibitem{R8}
Topological data analysis and machine learning, \\
Daniel Leykam \& Dimitris G. Angelakis,  \\
Advances in Physics: X, 8:1, 2202331, (2023)\\
DOI: 10.1080/23746149.2023.2202331

\bibitem{R9}
  Gemini for Google Workspace \\
{\tt https://services.google.com/\-fh/files/misc/gemini-for-google-workspace-prompting-guide-101.pdf}

\end{thebibliography}
%%-----------------------------------------------
\end{document}
